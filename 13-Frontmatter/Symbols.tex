%
% -----------------------------------------------------
% > > > > > >          R E A D M E          < < < < < <
% -----------------------------------------------------
%
% Die jeweiligen Abschnitte müssen auskommentiert werden 
% wenn kein Symbol daraus im Dokument verwendet wird. 
% Sonst erscheint die Fehlermeldung "missing \item"
%
%
% -----------------------------------------------------
% >>>> A N F A N G 
% 
% L A T E I N I S C H E   B U C H S T A B E N
% -----------------------------------------------------
%
\section*{Lateinische Buchstaben} 		% * heißt, dass keine Nummer vor die Überschrift kommt
%
\begin{acronym}[-------------------] 	% In eckigen Klammern den längsten Ausdruck der Liste einfügen für Einzug
	%	
	% Acronym Definition
	%					\acro{KÜRZEL}[ABKÜRZUNG]{\acrounit{SI-EINHEIT}BESCHREIBUNG}
	%
	%
	% - - - - - - - - - - - - - - - - - - - - - - - - - - -
	% >>> G R O ß B U C H S T A B E N
	% - - - - - - - - - - - - - - - - - - - - - - - - - - -
	%
    %
	% --------------------
	\acro{A} 	[$A$]				{\acrounit{\m\squared}
		%	Equivalent absorption area in a diffuse acoustic field
		Äquivalente Absoptionsfläche in einem diffusen Schallfeld
	} % DIN EN ISO 80000-8:2008-01, p.16
	% --------------------
	\acro{B}	[$B_{\mathrm{p}}$]	{\acrounit{\N\m}
		%	Bending stiffness per unit width of plate
		Biegesteifigkeit je Meter Plattenbreite
	}
	% --------------------
	\acro{DTF} 	[$D_\mathrm{TF}$]				{\acrounit{\decibel}
		%	Transmission function
		Übertragungsfunktion
	} % DIN EN ISO 10848-1:2018-2 Abschnitt 3.5
	% --------------------
	\acro{DTFav} 	[$D_\mathrm{TF,av}$]				{\acrounit{\decibel}
		%	Spatioal average transmission function
		Mittlere Übertragungsfunktion
	} % DIN EN ISO 10848-1:2018-2 Abschnitt 3.6
	% --------------------
	\acro{E}	[$E$]				{\acrounit{\N\per\m\squared}
		%	Young's modulus
		Elastizitätsmodul
	} % DIN EN ISO 80000-4:2013-08, p.14
	% --------------------
	\acro{Ep}	[$E_{\mathrm{p}}$]	{\acrounit{\joule}
		%	Potential energy
		Potentielle Energie
	} % DIN EN ISO 80000-4:2013-08, p.18
	% --------------------
	\acro{Ek}	[$E_{\mathrm{k}}$]	{\acrounit{\joule}
		%	Kinetic energy
		Kinetische Energie
	} % DIN EN ISO 80000-4:2013-08, p.18
	% --------------------
	\acro{F}	[$F$]				{\acrounit{\N}
		%	Force
		Kraft
	} % DIN EN ISO 80000-4:2013-08, p.8
	% --------------------
	\acro{G}	[$G$]				{\acrounit{\N\per\m\squared}
		%	Shear modulus
		Schubmodul
	} % DIN EN ISO 80000-4:2013-08, p.14
	% --------------------
	\acro{I}	[$\mathbf{I}$]		{\acrounit{\watt\per\m\squared}
		%	Intensity
		Intensität
	} % DIN EN ISO 80000-8:2008-01, p.12
	% --------------------
	\acro{K}	[$ K $]				{\acrounit{\pascal}
		%	Bulk modulus
		Kompressionsmodul
	} % DIN EN ISO 80000-4:2013-08, p.14
	% --------------------
	\acro{Kij}	[$ K_{ij} $]		{\acrounit{\dB}
		%	Vibration reduction index
		Stoßstellendämm-Maß zwischen Bauteil 
		$i$ und $j$
	}
	% --------------------
	\acro{La} 	[$L_a$] 			{\acrounit{\dB}
		%	Acceleration level with 
		Beschleunigungspegel mit
		$a_0=\SI{1e-6}{\m\per\s\squared}$
	} 		
	% --------------------
	\acro{Lav} 	[$L_\mathrm{av}$]				{\acrounit{\decibel}
		%	Transmission function
		Mittlerer Schalldruckpegel im Empfangsraum (ISO 10848-1:2017)
	} % DIN EN ISO 10848-1:2018-2 Abschnitt 3.5
	% -------------------- 
	\acro{LI} 	[$L_a$] 			{\acrounit{\dB}
		%	Intensity level with 
		Intensitätspegel mit
		$I_0=\SI{1e-12}{\watt}$
	} 		
	% -------------------- 
	\acro{Lp} 	[$L_p$] 			{\acrounit{\dB}
		%	Sound pressure level with 
		Schalldruckpegel mit	
		$p_0=\SI{20e-6}{\pascal}$
	} % DIN EN ISO 80000-8:2008-01, p.14
	% --------------------
	\acro{Ln}	[$L_{\mathrm{n}}$]	{\acrounit{\decibel}
		%	Normalized impact sound pressure level
		Norm-Trittschallpegel im Prüfstand
	}
	% --------------------
	\acro{Lne0f}	[$L_{\mathrm{ne0,f}}$]	{\acrounit{\decibel}
		%	Normalized flanking equipment sound pressure level
		Norm-Flankengeräteschallpegel
	} % ISO 10848-1:2017
	% --------------------
	\acro{LW} 	[$L_W$] 			{\acrounit{\dB}
		%	Sound power level with 
		Schallleistungspegel mit
		$P_0=\SI{1e-12}{\watt}$
	} % DIN EN ISO 80000-8:2008-01, p.14
	% --------------------
	\acro{Lv} 	[$L_v$] 			{\acrounit{\dB}
		%	Velocity level with 
		Schnellepegel mit
		$v_0=\SI{1e-9}{\m\per\s}$
	}
	% -------------------- 
	\acro{N}	[$N$]				{\acrounit{-}
		%	Modes in band
		Modenanzahl je Frequenzband
	}
	% --------------------
	\acro{P}	[$P$]				{\acrounit{\m}
		%	Perimeter
		Umfang
	} % ToDo ändern
	% --------------------
	\acro{P}    [$P$] 				{\acrounit{\watt}
		%	Power
		Leistung
	} % DIN EN ISO 80000-8:2008-01, p.14
	% --------------------
	\acro{R}	[$R$]				{\acrounit{\dB}
		%	Sound reduction index
		Schalldämm-Maß
	}% DIN EN ISO 80000-8:2008-01, p.16
	% --------------------
	\acro{Rij}	[$ R_{ij} $]		{\acrounit{\dB}
		Flanking sound reduction index between element   
		% Flankendämm-Maß zwischen Element 
		$i$ und $j$
	}
	% --------------------
	\acro{S}	[$S$]				{\acrounit{\m\squared}
		%	Area
		Fläche
	} % DIN EN ISO 80000-3:2013-08, p.10
	% --------------------
	\acro{T}	[$T$]				{\acrounit{\s}
		%	Period duration
		Periodendauer
	} % DIN EN ISO 80000-3:2013-08, p.8
	% --------------------
	\acro{T60}	[$T_{60}$]			{\acrounit{\s}
		%	Reverberation time, linear regression over a range of 
		Nachhallzeit, lineare Regression über einen Abfall von
		\SI{60}{\dB}
	} % DIN EN ISO 80000-8:2008-01, p.16
	% --------------------
	\acro{V}	[$V$]				{\acrounit{\m\cubed}
		%	Volume
		Volumen
	} % DIN EN ISO 80000-3:2013-08, p.10
	% --------------------
	\acro{W}	[$W$]				{\acrounit{\watt}
		%	Mechanic power
		Mechanische Leistung
	} % DIN EN ISO 80000-4:2013-08, p.18
	% --------------------
	\acro{Y}	[$Y$]				{\acrounit{\m\per\newton\s}
		%	Mobility
		Admittanz
	}
	% --------------------
	\acro{Zc}   [$Z_{\mathrm{c}}$] 	{\acrounit{\pascal\s\per\m}
		%	Characteristic impedance of a medium
		Charakteristische Impedanz eines Mediums 
	} % DIN EN ISO 80000-8:2008-01, p.12
	% --------------------
	\acro{Zm}   [$Z_{\mathrm{m}}$] 	{\acrounit{\newton\s\per\m}
		%	Mechanical impedance
		Mechanische Impedanz
	} % DIN EN ISO 80000-8:2008-01, p.12
	% --------------------
	%
	%
	%
	%
	%
	\bigskip 	% Größerer vertikaler Abstand
	%
	% - - - - - - - - - - - - - - - - - - - - - - - - - - -
	% >>> K L E I N B U C H S T A B E N
	% - - - - - - - - - - - - - - - - - - - - - - - - - - -
	%
	% --------------------
	\acro{a}	[$\mathbf{a}$]		{\acrounit{\m\per\s\squared}
		%	Acceleration
		Beschleunigung
	} % DIN EN ISO 80000-8:2008-01, p.10
	% --------------------
	\acro{a}	[$a$]				{\acrounit{\si{\m}}
		%	equivalent absorption length
		äquivalente Absorptionslänge
	}
	% --------------------
	\acro{c}	[$c$]				{\acrounit{\m\per\s}
		%	Phase velocity of propagation
		Ausbreitungsgeschwindigkeit der Phase einer Welle
	} % DIN EN ISO 80000-8:2008-01, p.10
	% --------------------
	\acro{cg}	[$c_{\mathrm{g}}$]	{\acrounit{\m\per\s}
		%	Group velocity of propagation
		Gruppengeschwindigkeit
	} % DIN EN ISO 80000-8:2008-01, p.10
	% --------------------
	\acro{d}	[$d$]				{\acrounit{\m}
		%	Distance
		Abstand
	} % DIN EN ISO 80000-3:2013-08, p.8
	% --------------------
	\acro{d}	[$d$]				{\acrounit{\m}
		%	Thickness
		Dicke
	} % DIN EN ISO 80000-3:2013-08, p.8
	% --------------------
	\acro{f}	[$f$]				{\acrounit{\Hz}
		%	Frequency
		Frequenz
	} % DIN EN ISO 80000-8:2008-01, p.8
	% --------------------
	\acro{fc}	[$f_{\mathrm{c}}$] 	{\acrounit{\Hz}
		%	Critical frequency
		Koinzidenzgrenzfrequenz
	}
	% --------------------
	\acro{k}	[$k$]				{\acrounit{\radian\per\m}
		%	Wave number
		Kreiswellenzahl
	} % DIN EN ISO 80000-8:2008-01, p.8
	% --------------------
	\acro{j}	[$\mathrm{j}$]		{\acrounit{-}
		%	Imaginary unit $\mathrm{j}^2=-1$
		Imaginäre Einheit $\mathrm{j}^2=-1$
	} % DIN EN ISO 80000-2:2013-08, p.21
	% --------------------
	\acro{m}	[$m$]				{\acrounit{\kg}
		%	Mass
		Masse
	} % DIN EN ISO 80000-4:2013-08, p.6
	% --------------------
	\acro{l}	[$l$]				{\acrounit{\m}
		%	Length
		Länge
	} % DIN EN ISO 80000-4:2013-08, p.8
	% --------------------
	\acro{lij}	[$ l_{ij} $]		{\acrounit{\m}
		%	Junction length between elements $ i $ and $ j $
		gemeinsame Kopplungslänge von Bauteil $i$ und $j$ 
	}
	% --------------------
	\acro{n-2D}	[$n^{\mathrm{2D}}$]	{\acrounit{\radian\per\s}
		%	Modal density of a plate
		Modale Dichte einer Platte
	}
	% --------------------
	\acro{n-3D}	[$n^{\mathrm{3D}}$]	{\acrounit{\radian\per\s}
		%	Modal density of a volume
		Modale Dichte eines Volumens
	}
	% --------------------
	\acro{p}	[$p$]				{\acrounit{\pascal}
		%	Sound pressure
		Schalldruck
	} % DIN EN ISO 80000-8:2008-01, p.10
	% --------------------
	\acro{ps}	[$p_\mathrm{s}$]	{\acrounit{\pascal}
		%	Static pressure
		Statischer Druck
	} % DIN EN ISO 80000-8:2008-01, p.10
	% --------------------
	\acro{t}	[$t$]				{\acrounit{\s}
		%	Time
		Zeit
	} % DIN EN ISO 80000-3:2013-08, p.12
	% --------------------
	\acro{v}	[$\mathbf{v}$]		{\acrounit{\m\per\s}
		%	Velocity
		Schallschnelle
	} % DIN EN ISO 80000-8:2008-01, p.10
	% --------------------
	%
	%
\end{acronym}
%
% -----------------------------------------------------
% L A T E I N I S C H E   B U C H S T A B E N
% 
% >>>> E N D E
% -----------------------------------------------------
%
%
%
%
%
%
%
% -----------------------------------------------------
% >>>> A N F A N G
% 
% G R I E C H I S C H E   B U C H S T A B E N
% -----------------------------------------------------

\section*{Griechische Buchstaben}
% 
%
\begin{acronym}[-------------------] 	% In eckigen Klammern den längsten Ausdruck der Liste einfügen für Einzug
	%	
	% Acronym Definition
	%					\acro{KÜRZEL}[ABKÜRZUNG]{\acrounit{SI-EINHEIT}BESCHREIBUNG}
	%
	%
	%
	% - - - - - - - - - - - - - - - - - - - - - - - - - - -
	% >>> K L E I N E   G R.  B U C H S T A B E N
	% - - - - - - - - - - - - - - - - - - - - - - - - - - -
	%
	% --------------------
	\acro{alpha}	[$\alpha$]		{\acrounit{-}
		%	Absorption coefficient
		Absorptionsgrad
	} % DIN EN ISO 80000-8:2008-01, p.16
	% --------------------
	\acro{alphaD}	[$\alpha$]		{\acrounit{\neper\per\m}
		%	Attenuation coeffient
		Dämpfungskoeffizient
	} % DIN EN ISO 80000-8:2008-01, p.16
	% --------------------
	\acro{lambda}	[$\lambda$]		{\acrounit{\m}
		%	Wave length
		Wellenlänge
	} % DIN EN ISO 80000-8:2008-01, p.8
	% --------------------
	\acro{gamma}    [$\gamma$]  	{\acrounit{-}
		%	Shear strain
		Scherverformung
	} % DIN EN ISO 80000-4:2013-08, p.12
	% --------------------
	\acro{mu}		[$\mu$]			{\acrounit{-}
		%	Poisson's ratio
		Querkontraktionszahl
	} % DIN EN ISO 80000-4:2013-08, p.14
	% --------------------
	\acro{rho}		[$\rho$]		{\acrounit{\kg\per\m\cubed}
		%	Density
		Rohdichte
	} % DIN EN ISO 80000-8:2008-01, p.8
	% --------------------
	\acro{rhoA}		[$\rho_A$]		{\acrounit{\kg\per\m\squared}
		%	Mass per unit area
		Flächenbezogene Masse
	} % DIN EN ISO 80000-4:2013-08, p.6, cf. % DIN EN ISO 80000-3:2013-08, p.10
	% --------------------
	\acro{rhol}		[$\rho_l$]		{\acrounit{\kg\per\m}
		%	Mass per unit length
		Längenbezogene Masse
	} % DIN EN ISO 80000-4:2013-08, p.6
	% --------------------
	\acro{sigma}    [$\sigma$]  	{\acrounit{-}
		%	Radiation efficiency
		Abstrahlgrad
	}
	% --------------------
	\acro{sigma}    [$\sigma$]  	{\acrounit{\newton\per\meter\squared}
		%	Principal stress
		Mechanische Normalspannung
	} % DIN EN ISO 80000-4:2013-08, p.12
	% --------------------
	\acro{tau}		[$\tau$]		{\acrounit{-}
		%	Transmission coefficient
		Transmissionsgrad
	} % DIN EN ISO 80000-8:2008-01, p.16
	% --------------------
	\acro{tau}		[$\tau$]		{\acrounit{\newton\per\m\squared}
		%	Shear stress
		Schubspannung, Scherspannung
	} % DIN EN ISO 80000-4:2013-08, p.12
	% --------------------
	\acro{theta}	[$\theta$]		{\acrounit{\kelvin}
		%	Temperature
		Temperatur
	} % DIN EN ISO 80000-5:2013-08, p.6
	% --------------------
	\acro{kappa}	[$\kappa$]		{\acrounit{-}
		%	Shear correction factor
		Brechzahl
	} % ToDo Namen ändern
	% --------------------
	\acro{eta}		[$\eta$]		{\acrounit{-}
		%	Loss factor
		Verlustfaktor
	}
	% --------------------
	\acro{phi}		[$\phi$]		{\acrounit{\radian}
		%	Angular dimension
		Winkelmaß
	}
	% --------------------
	\acro{varphi}	[$\varphi$] 	{\acrounit{-}
		%	Relative humidity
		Relative Luftfeuchte
	} % DIN EN ISO 80000-5:2013-08, p. 19
	% --------------------
	\acro{omega}	[$\omega$]		{\acrounit{\radian\per\s}
		%	Angular frequency
		Kreisfrequenz	
	} % DIN EN ISO 80000-3:2013-08, p.8
	% --------------------
	%
	%
	%
	%
	%
	\bigskip 	% Größerer vertikaler Abstand
	%
	% - - - - - - - - - - - - - - - - - - - - - - - - - - -
	% >>> G R O ß E   G R.  B U C H S T A B E N
	% - - - - - - - - - - - - - - - - - - - - - - - - - - -
	%
	% --------------------
	\acro{gamma'bc}	[$\Gamma'_{\mathrm{BC}}$]	{\acrounit{-} 
		Factor for boundary condition
	}
	% --------------------
	\acro{gamma'1}	[$\Gamma'_{\mathrm{1}}$]	{\acrounit{-} 
		Factor for boundary condition
	}
	% --------------------
	\acro{gamma'2}	[$\Gamma'_{\mathrm{2}}$]	{\acrounit{-} 
		Factor for boundary condition
	}
	% --------------------
	%
	%
\end{acronym}
%
% -----------------------------------------------------
% G R I E C H I S C H E   B U C H S T A B E N
%
% >>>> E N D E
% -----------------------------------------------------
%
%
%
%
%
%
%
%
%
%
%
% -----------------------------------------------------
% >>>> A N F A N G
%
% T I E F S T E L L U N G E N
% -----------------------------------------------------
%
\section*{Tief- und Hochstellungen} 				% * heißt, dass keine Nummer vor die Überschrift kommt
%
\begin{acronym}[------] 					% In eckigen Klammern den längsten Ausdruck der Liste einfügen für Einzug
	%	
	% Acronym Definition
	%					\acro{KÜRZEL}[ABKÜRZUNG]{\acrounit{SI-EINHEIT}BESCHREIBUNG}
	%
	%
	% - - - - - - - - - - - - - - - - - - - - - - - - - - -
	% >>> T I E F S T E L L U N G E N
	% - - - - - - - - - - - - - - - - - - - - - - - - - - -
	%
	% --------------------
	\acro{subAir}	[$_{\mathrm{air}}$]{
		%	Air
		Luft
	}
	% --------------------
	\acro{subB}		[$_{\mathrm{B}}$]{
		%	Bending
		Biegung
	}
	% --------------------
	\acro{subc}     [$_{\mathrm{c}}$]{
		% Coincidence
		Koinzidenz
	}
	% --------------------
	\acro{subdp}	[$_{\mathrm{dp}}$]{
		Driving-point
	}
	% --------------------
	\acro{seff}		[$_{\mathrm{eff}}$]{
		%	Effective
		Effektiv
	}
	% --------------------
	\acro{subint}	[$_{\mathrm{int}}$]{
		Intern
	}
	% --------------------
	\acro{sL}		[$_{\mathrm{L}}$]{
		Longitudinal
	}
	% --------------------
	\acro{sublab}	[$_{\mathrm{lab}}$]{
		Labor situation
	}
	% --------------------
	\acro{subn}	[$_{\mathrm{n}}$]{
		% Normalized to absorption area
		Normiert auf Bezugsabsorptionsfläche \acs{A}
	}
	% --------------------
	\acro{subrad}	[$_{\mathrm{rad}}$]{
		%	Radiation
		Abstrahlung
	}
	% --------------------
	\acro{sR}		[$_{\mathrm{R}}$]{
		Rayleigh surface wave
	}
	% --------------------
	\acro{subR}		[$_{\mathrm{R}}$]{
		% Receiver
		Empfangsstruktur (Receiver)
	}
	% --------------------
	\acro{subRR}    [$_{\mathrm{RR}}$]  {
		%	Recieving room
		Empfangsraum
	}
	% --------------------
	\acro{subsitu}	[$_{\mathrm{situ}}$]{
		In situ
	}
	% --------------------
	\acro{subSR}    [$_{\mathrm{SR}}$]  {
		%	Sending room
		Senderaum
	}
	% --------------------
	\acro{subS}		[$_{\mathrm{S}}$]{
		% Source
		Quelle (Source)
	}
	% --------------------
	\acro{subtot}	[$_{\mathrm{tot}}$]{
		Total
	}
	% --------------------
	\acro{sT}		[$_{\mathrm{T}}$]{
		Transversal
	}
	% --------------------%
	%
	%
	%
	%
	\bigskip 	% Größerer vertikaler Abstand
	%
	% - - - - - - - - - - - - - - - - - - - - - - - - - - -
	% >>> H O C H S T E L L U N G E N
	% - - - - - - - - - - - - - - - - - - - - - - - - - - -
	%
	\acro{Prime}		[$^\prime$]{
		In-situ (inklusive aller Flankenwege)
	}
	% --------------------%
	%
	%
\end{acronym}

% -----------------------------------------------------
% T I E F S T E L L U N G E N
%
% >>>> E N D E
% -----------------------------------------------------
%
%
%
%
%
%
%
%
%
%
%
% -----------------------------------------------------
% >>>> A N F A N G
%
% M A T H.   O P E R A T O R E N
% -----------------------------------------------------
%
\section*{Mathematische Operatoren} 				% * heißt, dass keine Nummer vor die Überschrift kommt
%
\begin{tabular}{ll}
$x$                         & Platzhalter\\
$\underline{x}$             & Komplexer Wert\\
% $\underline{x}^\ast$        & komplex konjugiert\\
% $\vert x \vert$             & Betrag\\
% $\Re\{x\}$                  & Realteil\\
% $\Im\{x\}$                  & Imaginärteil\\
% $\hat{x}$                   & Spitzenwert\\
% $\tilde{x}$                 & Effektivwert\\
% $\langle x \rangle_{t}$     & zeitlicher Mittelwert\\
% $\langle x \rangle_{s}$     & räumlicher Mittelwert\\
% $\langle x \rangle_{t,s}$   & zeitlich und räumlicher Mittelwert
\end{tabular}
%
%
%
% -----------------------------------------------------
% M A T H.   O P E R A T O R E N
%
% >>>> E N D E
% -----------------------------------------------------